\documentclass{article}
\usepackage{amsmath, amssymb, amsthm}
\usepackage{fancyhdr}
\usepackage{lipsum} % for generating dummy text, you can remove this line in your actual document
\usepackage[margin=1in, bottom=1.5in]{geometry} % Adjust bottom margin as needed
\usepackage{thmtools}
\usepackage{listings}
\usepackage{graphicx}
\usepackage{hyperref}
\usepackage{esint}
\usepackage{circuitikz}
\usepackage{chemformula}

% Page style settings
\pagestyle{fancy}
\fancyhf{} % Clear header and footer
\renewcommand{\headrulewidth}{1pt}
\renewcommand{\footrulewidth}{1pt}
\renewcommand{\labelenumi}{(\alph{enumi})}
\fancyhead[C]{\textbf{\large 1A - Structures and Materials}}
\fancyfoot[C]{\thepage}

\DeclareMathOperator{\tr}{Tr}

% Macros for convenience
\newcommand{\bbR}{\mathbb{R}} % Example: Real numbers
% Add more macros as needed

% Define theorems, propositions, definitions, etc. using thmtools
\declaretheoremstyle[
  spaceabove=6pt,
  spacebelow=6pt,
  headfont=\bfseries,
  notefont=\normalfont,
  bodyfont=\normalfont,
  headpunct={},
  postheadspace=1em,
  qed=,
]{mystyle}

\declaretheorem[
  style=mystyle,
  name=Theorem,
  within=section,
]{theorem}

\declaretheorem[
  style=mystyle,
  name=Proposition,
  within=section,
]{proposition}

\declaretheorem[
  style=mystyle,
  name=Definition,
  within=section,
]{definition}

\declaretheorem[
  style=mystyle,
  name=Example,
  within=section,
]{example}

\begin{document}

\title{Engineering Tripos Part IA - Structures and Materials}
\author{Morărescu Mihnea-Theodor}
\date{\today}

\maketitle

\newpage

\tableofcontents

\newpage

\section{Structures}

Structural mechanics aims to provide stiffness in useful locations. Stiffness, the ratio of force to displacement, is never infinite, as we have no absolutely rigid materials, and never zero, as we always face friction or drag. In many applications, the loads on structures are primarily due to weight. However, dynamic systems always require forces to accelerate or decelerate their components and these may far exceed the loads due to gravity.

\subsection{External forces}

Forces may be classified as either external (e.g. applied) forces, or internal forces. External forces may cause acceleration, and are always resisted by internal forces within the structure. The structure will deform in response to these internal forces. The logic of the first term's course is to first understand external forces, then to translate these into internal forces, and then to explore how the structure deforms.

\subsubsection{Equilibrium}

\begin{proposition}[Equilibrium conditions]
    For a structure to be in equilibrium, we must impose that the resultant force acting on the body is zero, and the resultant moment about any point is also null, i.e.:

    \[ \sum_{i} \mathbf{F}_i = \mathbf{0} \text{   and   } \sum_{i} \mathbf{M}_i = \mathbf{0} \]
\end{proposition}

The two above equations imply that the vector polygon (a skew polygon) of all forces, and the polygon of all moments acting on the body are closed.

\begin{figure}[h]
    \centering
    \includegraphics[width = 0.3\textwidth]{images/force_polygon.png}
    \caption{Force polygon}
    \label{fig:force-polygon}
\end{figure}

\begin{example}
    The graphical approach for equilibrium is particularly strong because of the sine theorem. Consider the example of a weight that is supported by a string in the following manner.
    
    \begin{figure}[h]
    \centering
    \includegraphics[width = 0.75\textwidth]{images/ex1.png}
    \caption{Three force equilibrium problem}
    \label{fig:ex-1}
    \end{figure}

    Because, of this, we can draw the following force polygon:

    \begin{figure}[h]
    \centering
    \includegraphics[width = 0.25\textwidth]{images/ex_2.png}
    \caption{Three force equilibrium problem - graphical solution}
    \label{fig:ex-2}
    \end{figure}

    The weight $\mathbf{W}$ acts vertically down, the tension $\mathbf{T}$ acts at an angle $\alpha$, and the force $\mathbf{F}$ acts at an angle $\frac{\pi}{2} - \beta$. By means of the sine theorem, we deduce that:

    \[ \frac{F}{\sin{\alpha}} = \frac{T}{\cos{\beta}} = \frac{W}{\cos{(\beta - \alpha)}} \]
\end{example}

Moreover, if a three-force member is in equilibrium and the forces are not parallel, they must be concurrent. Therefore, the lines of action of all three forces acting on such a member must intersect at a common point. Any single force is therefore the equilibrant of the other two forces. If there are more than three forces acting on a body, we can combine any two of them to further reduce the problem to only three forces.

\subsubsection{Distributed loads}

In reality, point forces cannot exist - as force is equal to pressure times area, a point force would require an infinite pressure (which does not make sense).

\begin{figure}[h]
    \centering
    \includegraphics[width = 0.5\textwidth]{images/press.png}
    \caption{Point forces are in reality distributed}
    \label{fig:distributed_loads}
\end{figure}

All forces are therefore distributed over a finite area. When this area is relatively small, descriptions based on point forces can helpfully simplify the analysis without much loss of accuracy. However, some forces such as those applied to a body by fluids or gravitational attraction are distributed over a large enough area that we must account for their distribution in our analysis.

\subsubsection{Constant fluid pressure}

The pressure imposed on structures by stationary fluids is significant in applications such as pressure vessels or storage tanks.

\begin{theorem}[Pascal's law]
    The pressure $p$ at any given point in a stationary fluid is the same in all directions. Where the fluid meets a constraining surface, over any infinitesimal area, it imparts a force $\delta F = p \delta S$ normally to the surface.
\end{theorem}

Therefore, if our surface is a curve with unit normal vector $\mathbf{N}$, then:

\[ d\mathbf{F} = p (d\mathbf{S} \cdot \mathbf{N}) \mathbf{N} \]

However, if we have a curve, we need to perform a surface integration in order to obtain the total force acting upon it - this will be covered in the Structures course in part IB. We can, however, simplify our calculations by considering surfaces of unit width, and hence:

\[ \mathbf{F} = pL\mathbf{N} \]

\begin{example}
    For instance, let us consider the case of a curved surface in the shape of a semi-circle of unit width with a uniform pressure acting upon it. Then, for an infinitesimally small element of length:

    \[ dF = prd\theta \]

    And by means of integration:

    \[ F = \int_0^\pi prd\theta = \pi pr \]
\end{example}

\subsubsection{Hydrostatic loading}

Only in rare cases are structures loaded by a strictly uniform pressure - typically only in high pressure chambers. A more common form of pressure loading arises due to the weight of the fluid, which varies with depth and is known as hydrostatic pressure.

By means of equilibrium for a column of liquid, we can equate the Archimedic force with the force due to pressure:

\[ p(h)S = \rho Vg \]

This means that:

\[ p(h)S = \rho Sgh \iff p(h) = \rho gh \]

Now, consider the case where we have to replace a hydrostatic load from a height $h_1$ to a height $h_2$, with $h_1 < h_2$ with a point force. First, we can calculate the equivalent force. Suppose the area is of unit width. Then:

\[ F = \int_{h_1}^{h_2} \rho ghdh = \frac{1}{2}\rho g(h_2^2 - h_1^2) \]

And the equivalent point is calculated by taking moments about any of $h_1$ or $h_2$:

\[ Fh^* = \int_{h_1}^{h_2}\rho g h^2 dh \]

Implying that:

\[ h^* = \frac{\int_{h_1}^{h_2} \rho gh^2dh}{\int_{h_1}^{h_2} \rho g h dh} \]

\subsubsection{Gravity}

All structures are subject to forces due to gravity, whether through weights loading a structure or due to the self-weight of its components. These forces are always distributed, but to simplify analysis can often be treated as point forces at a particular location. We will not go into the same depth of analysis as in the Mechanics course. However, using the same result from it, we can see that:

\[ (x, y) = \left(\frac{\sum_i m_ix_i}{\sum_i m_i}, \frac{\sum_i m_iy_i}{\sum_i m_i}\right) \]

For a general lamina with uniform density and thickness, we know that $dm = \rho t dS$. The above equation can be thus restated as:

\[ (x, y) = \left(\frac{\int xdS}{\int dS}, \frac{\int ydS}{\int dS}\right) \]

Now, consider a general distributed force per length given by the function $f(x)$. We can then deduce, that in the most general case, the equivalent force is equal to:

\[ F = \int f(x)dx \]

And the equivalent point of application can be determine by taking moment:

\[ Fx^* = \int f(x)xdx \iff x^* = \frac{\int xf(x)dx}{\int f(x)dx} \]

\subsubsection{Contact forces}

Contact forces are reaction forces - knowing that two bodies are in contact tells us something about their displacement but gives us no information about the magnitude of the force between them. We can only find the contact force as the reaction to other applied forces, and the distribution and extent of these forces may vary over the history of loading as the contacting bodies interact.

In frictionless contact, a distributed force acts equally and oppositely on both bodies, similarly to the effect of static pressure on a body in previous subsections. Firctionless contact forces act normally to the contact interface, as shown below.

\begin{figure}[h]
    \centering
    \includegraphics[width = 1\textwidth]{images/fcontact.png}
    \caption{The contact interface and common normal in frictionless contact}
    \label{fig:frictionless-contact}
\end{figure}

However, in reality we always have friction - this gives birth to a friction force $F = \mu N$ that acts normally to the common normal and that opposes movement. In structures, we will ditch the idea of friction being two forces, and instead we will introduce a reaction force $\mathbf{R}$ that represents the combined effects of the normal reaction and the friction force. Again, if there is no friction, we just have the common normal. If we have friction, the reaction starts to rotate by a certain angle. We distinguish between two cases:

\begin{enumerate}
    \item For static contact (no sliding), we must have that $F \leq \mu_sN$
    \item For dynamic contact (sliding between the bodies), we must have that $F = \mu_dN$
\end{enumerate}

Note that the static coefficient of friction is always higher than the dynamic coefficient of friction, i.e.:

\[ \mu_s > \mu_d \]

\begin{definition}[Angle of friction]
    The angle of friction is defined as:
    
    \[ \tan{\phi} = \frac{F}{N} \]
\end{definition}

Note that at any given point, it is true that the angle of friction is always less or equal to the static angle of friction, i.e.:

\[ \phi \leq \phi_s \]

There are different types of problems involving friction. If we have static contact, without impending motion, $F$ is not proportional to $N$, so $F$ must be found from consideration of equilibrium of the body only. However, it is true that $F < \mu_sN$ and $\phi < \phi_s$. If we have a static contact, on point of slipping problem, the angle of friction is equal to the angle of static friction, i.e. the friction force reaches is maximum value $F = \mu_sN$.

\subsubsection{Distributed friction}

\begin{example}[Sliding carpet]
    Find the force $T$ required to pull a carpet of length $L$ and weight $\omega$ per unit length at steady speed over a floor with coefficient of dynamic friction $\mu_d$.

    By considering equilibrium of a small section of carpet we deduce:

    \[ T(x) + dT = T(x) + \mu_dN \]
    \[ N = \omega dx \]

    Combining the two, we obtain:

    \[ dT = \mu_d \omega dx \iff T(x) = \mu_d \omega x + C\]

    We can determine the constant $C$ because we know that $T(0) = 0 \iff C = 0$. Hence:

    \[ T(x) = \mu_d \omega x \iff T = \mu_d \omega L \]
\end{example}

\begin{example}[Flexible belt on a pulley]
    A flexible belt is passed over a pulley as shown in the figure below. If the coefficient of static friction between the belt and the pulley is $\mu$, and $T_2 > T_1$, find the maximum ratio of $\frac{T_2}{T_1}$ before sliding occurs.

    \begin{figure}[h]
        \centering
        \includegraphics[width = 0.75\textwidth]{images/pulley.png}
        \caption{Belt passed over a pulley}
        \label{fig:pulley}
    \end{figure}

    By means of vertical equilibrium, we determine that:

    \[ R = 2T\sin{\frac{d\theta}{2}} + dT\sin{\frac{d\theta}{2}} \]

    Neglecting the terms with two differentials, and using the fact that $\sin{x} \to x$, as $x \to 0$, we deduce that:

    \[ R = Td\theta \]

    Horizontal equilibrium gives us:

    \[ dT = \mu R \]

    By combining both relationships:

    \[ dT = \mu T d\theta \iff \frac{1}{T} dT = \mu d\theta \]

    Denoting $\psi$ as the angle the belt moves over the pulley:

    \[ \int_{T_1}^{T_2} \frac{1}{T}dT = \int_0^\psi \mu d\theta \]

    Conversely:

    \[ \ln{\frac{T_2}{T_1}} = \mu\psi \]

    By exponentiating the result above, we deduce that:

    \[ \frac{T_2}{T_1} = e^{\mu\psi} \]

    This is a well known result, typically stated as:

    \[ \frac{T_{\text{big}}}{T_{\text{small}}} = e^{\mu\psi} \]
\end{example}

\subsubsection{Pin-joints}

Pin-joints are often used to connect structures to foundations. If the pin is frictionless, then the joint cannot apply a moment to the structures, which can be an advantage for structures such as trusses. A pin-joint applies a reaction force to the structure. As a result, like the contact forces of the previous section, it is only possible to find the magnitude and direction of the reaction at the pin-joint by solving the conditions of equilibrium in response to a set of applied loads. However, in contract with contact forces, the reaction force at a pin-joint can act in any direction.

\begin{figure}[h]
    \centering
    \includegraphics[width = 0.75\textwidth]{images/pinjoint.png}
    \caption{Pin-joint able to apply a reaction force in any given direction}
    \label{fig:enter-label}
\end{figure}

In analysing problems with pin-joints, it is assumed that they are frictionless and apply a point force to the structure. For pin joints, they restrict two degrees of freedom (they have both a vertical and a horizontal reaction).

\subsubsection{Roller supports}

A general three-dimensional body has six degrees-of-freedom: it would require six scalar
parameters to describe its motion, three linear translations and three rotations. The two-dimensional bodies considered in this course have three degrees-of-freedom. They can translate
horizontally or vertically (or in any other two non-parallel directions) and they can rotate.
Therefore, in order to ensure that a two-dimensional body, or structure, remains stationary it must
have three independent constraints.

However, were the beam supported by two pin-joints, as illustrated before, it would be over-constrained. The two pin-joints provide four constraints but the body has only three degrees of
freedom. As a result, it is not possible to find the reaction forces at the supports just be analysis of
equilibrium. To find all four reactions, we would need to know about the stiffness of the beam
and the supports, to find out how they interact. This is possible, but apart from complicating the
analysis, over-constraining a structure may have other unintended consequences. For example, if
a steel railway bridge is over-constrained, then thermal expansion or contraction of the bridge in
summer and winter would create additional and unwanted loading. 

For many structures, it is therefore important not to provide too many restraints. Therefore, if one
end of a beam is supported on a pin-joint, the other must have a different form of support. A
common solution is shown below a pin-joint is mounted on frictionless rollers. This
provides a vertical restraint on displacement but does not inhibit horizontal movement. Exactly
as with frictionless contact in the previous session, this form of support leads to a reaction force
in the direction of the common normal (i.e. perpendicular to the motion of the rollers). It is now
possible to solve the equilibrium of the beam just as we previously did it.

\begin{figure}[h]
    \centering
    \includegraphics[width = 0.75\textwidth]{images/roller.png}
    \caption{\centering A beam on pin-joint supports: (a) over-constrained (b) with one pin-joint on rollers (c) with a double sided pin-joint on rollers}
    \label{fig:enter-label}
\end{figure}

\subsubsection{Built-in/"encastré" supports}

A roller-support constrains one degree of freedom of the displacement of a structure, while a pin-joint or simple support constrains two. The third important form of support therefore is one that
constraints all three degrees of freedom: linear displacement in either direction and rotation. This
is achieved by “building-in” the structure to the ground or some other reference at an “encastré”
(literally “en-cased”) support as illustrated below. This support – found in cantilevered
structures such as balconies or loading cranes that project out of the side of buildings, or in the
foundations of many concrete buildings – constraints all three degrees of freedom of a two-dimensional structure, so fully enforces equilibrium. The reaction forces provided by an encastré
support include two components of a force and a moment in reaction to the constraint on rotation.

\begin{figure}[h]
    \centering
    \includegraphics[width = 0.5\textwidth]{images/encastre.png}
    \caption{(a) horizontal and (b) vertical encastré supports}
    \label{fig:enter-label}
\end{figure}

\newpage

\subsection{Internal forces}

We will now focus our attention towards exploring how external forces are carried by structures.

\subsubsection{Pin-jointed trusses}

\begin{definition}[Pin-jointed truss]
    A framework of members joined at their ends is called a truss. A pin-jointed truss is a truss made up of struts that are joined at their ends by pin-joints.
\end{definition}

In this course we shall only consider a particular type of trusses, where all of the members lie in one plane - such frameworks are called two-dimensional/plane trusses.

\begin{figure}[h]
    \centering
    \includegraphics[width = 0.75\textwidth]{images/trusses.png}
    \caption{Example plane trusses}
    \label{fig:enter-label}
\end{figure}

We will make a few more assumptions before proceeding to analyzing trusses:

\begin{enumerate}
    \item We shall consider one type of connection only - the idealized frictionless pin, thus leading to pin-jointed plane trusses
    \item We will consider pin-jointed trusses subject to point loads applied at the joints only
    \item We assume straight, slender members
\end{enumerate}

Moreover, because the member is straight, if it were cut anywhere along its length, the internal force must equal the applied external force and is aligned along the members - exactly as if it were a string under tension. Thus, each members is in a state of pure tension or pure compression, i.e. it is an axial force member.

\subsubsection{Statically determinate frames}

In introducing the design of supports in a previous section of this course, we noted that structures should never be under- or over-constrained. For the two-dimensional structures we are considering in this course, they require three independent external restraints in order to remain fixed (the restraints must prevent rotation and translation in two directions). A similar issue arises with the design of the internal components of the structure: if they are under-connected, the structure will be a mechanism and may collapse without the application of any load; if they are over-connected, this may be a sign of inefficiency, but the analysis of the structure will be more difficult, as it is now statically indeterminate.

\begin{proposition}[Euler's conditions for structures]
    Let us consider a structure where $D$ is the number of dimensions, $j$ is the number of joints, $b$ is the number of bars, and $r$ is the number of restraints imposed upon the structure. Therefore:

    \begin{enumerate}
        \item If $Dj > b + r$, then the structure is a mechanism
        \item If $Dj = b + r$, then the structure is statically determinate
        \item If $Dj < b + r$, then the structure is statically indeterminate
    \end{enumerate}
\end{proposition}

However, this formula may fail in certain cases, such as the one below.

\begin{figure}[h]
    \centering
    \includegraphics[width = 0.25\textwidth]{images/indet.png}
    \caption{Fail point of Euler's conditions}
    \label{fig:enter-label}
\end{figure}

The structure above is made of a mechanism and a statically indeterminate truss. Remember that for the formula to work, we need to only have our truss made out of triangles that do not overlap.

\subsubsection{Method of joints}

The method joints requires that we create a free-body diagram for each pin-joint of the truss. It provides a systematic way of calculating all the bar forces in a structure, purely by considering equiilibrium conditions at each point.

\begin{example}[Example usage of the method of joints]
    Consider the truss shown below. 

    \begin{figure}[h]
    \centering
    \includegraphics[width = 0.25\textwidth]{images/truss1.png}
    \caption{Method of joints example}
    \label{fig:enter-label}
    \end{figure}

    We begin by assuming all members are in tension, with the convention that tension is positive and compression is negative. By vertical equilibrium at $A$:

    \[ T_{\text{AB}}\frac{\sqrt{2}}{2} = W \iff T_{\text{AB}} = W\sqrt{2}\]

    Horizontal equilibrium gives:

    \[ W + T_{\text{AC}} = 0 \iff T_{\text{AC}} = -W \]

    Therefore, $AC$ is a compression member, while $AB$ is a tension member.
\end{example}

The method of joints is the better alternative when we need to find the forces in all members of a truss. However, there is another method if we need to find the tension in only a few members.

\subsubsection{Method of sections}

If we need to know the force in only one or two members of a truss, and the members are far from the supports, the method of joints involves a lot of work. In such cases, the method of sections can save a lot of effort.

In order to find the axial force in one specific bar only, we need to:

\begin{enumerate}
    \item Consider a continuous cut through the truss - it is useful to make the cut so that we only reveal a maximum of three members
    \item Consider the part of the truss that has less forces (external) acting upon it
    \item Reveal the forces and take moments such that two of the forces cancel
\end{enumerate}

\begin{example}
    Consider the plane pin-jointed truss below. Find the axial forces in bars $DF$ and $EG$.

    \begin{figure}[h]
    \centering
    \includegraphics{images/truss3.png}
    \caption{Method of sections for a plane truss}
    \label{fig:enter-label}
    \end{figure}

    Consider the cut above. To find the reaction through $EG$, take moments about $F$ after revealing the three forces in the members:

    \[ T_{\text{EG}}L + 3WL = 0 \iff T_{\text{EG}} = -3W \]

    Now, to find the tension in the member $DF$, take moments about $E$:

    \[ T_{\text{DF}}L = 2WL \iff T_{\text{DF}} = 2W \]
\end{example}

Furthermore, there are three main simplifications we can use when analyzing pin-jointed trusses. Let us consider the figure below.

\begin{figure}[h]
    \centering
    \includegraphics{images/truss4.png}
    \caption{Special cases}
    \label{fig:enter-label}
\end{figure}

For the first pin, $T_3 = 0$ and $T_1 = T_2$. For the second joint, because there is no external force applied, $T_1 = T_2 = 0$. For the last pin-joint, $T_1 = T_3$ and $T_2 = T_4$.

In general, when computing forces in all members of a truss, we first begin by calculating the external reactions, and then proceeding either by the method of sections or the method of joints. Consider the truss structure below.

\begin{example}
    Compute the force in the member $DE$ due to a vertical load $W$ at $E$.

    \begin{figure}[h]
    \centering
    \includegraphics[width = 0.4\textwidth]{images/truss5.png}
    \caption{Truss with a single vertical load}
    \label{fig:enter-label}
    \end{figure}

    Since $A$ is pin-jointed, there will be two reaction forces $V_A$ and $H_A$. $B$ is a roller support, so we only have a vertical reaction $V_B$. By horizontal equilibrium, $H_A = 0$. Taking moments about $A$:

    \[ WL = 4V_BL \iff V_B = \frac{W}{4} \]
 
    By means of vertical equilibrium, the reaction at $A$ is given by:

    \[ V_A = W - \frac{W}{4} = \frac{3W}{4} \]

    Consider a cut going through $DE$, $EC$ and $AC$ and take the left side of the truss. Take moments about $C$:

    \[ T_{\text{DE}} L\sqrt{2} + 2V_BL = 0 \iff T_{\text{DE}} = -W\frac{\sqrt{2}}{4} = -\frac{W}{2\sqrt{2}} \]
\end{example}

\subsubsection{Superposition}

Consider a plane truss to which we apply a sequence of loads $W_1, W_2, \dots, W_n$. Let us consider that there exists constant $\lambda_k$ so that upon imposing the load $W_k$, the axial force in a chosen member is given by $\lambda_kW_k$. Therefore, the force $T$ due to applying all loads at once is given by:

\[ T = \sum_{k = 1}^n \lambda_kW_k \]

Hence, plane trusses form a linear system. Furthermore, the conditions upon the behavior of a truss is linear are:

\begin{enumerate}
    \item The material of the truss remains in the linear elastic range
    \item The geometry changes, i.e. the distortion of the structure caused by the loads is small - hence, the equilibrium equations written in the undeformed configuration are also valid after the structure has deformed
\end{enumerate}

Note that these conditions are inherently true for most structures made out of metal.

\subsubsection{Symmetry}

Many structures are symmetric - both for aesthetic reasons and to simplify their production. Symmetry always leads to a reduction in the effort of analyzing a structure, and therefore should be always exploited.

A common example of symmetry for trusses is when they are symmetric about a vertical centre-line, and are said to have "mirror symmetry". If a truss has symmetry about a line, and the loads are symmetric about the same line, then it follows that the bar tensions (and support reactions) must also be symmetric about that line.

\begin{figure}[h]
    \centering
    \includegraphics{images/symmetry.png}
    \caption{Symmetric loading on a truss leads to symmetric internal forces}
    \label{fig:enter-label}
\end{figure}

Furthermore, for a symmetric structure it is always possible, and sometimes even convenient, to represent any set of applied loads by superposing a symmetryc and an anti-symmetric set of loads, as shown below:

\begin{figure}[h]
    \centering
    \includegraphics{images/symmetry2.png}
    \caption{Symmetric/anti-symmetric load combination}
    \label{fig:enter-label}
\end{figure}

\subsubsection{Shear forces and bending moments}

The previous sections explored how external forces acting on a pin-jointed planar truss can be translated into internal forces. The assumption of frictionless pin-joints and the use of only straight members and nodal loading ensured that all the members of the truss were axially loaded, so the internal forces must be tension or compression aligned with the member. The figure below shows that if any of these requirements is not met, then when a free-body diagram is created for a section of the member, it cannot be held in equilibrium by a single axial force.

\begin{figure}[h]
    \centering
    \includegraphics{images/shear1.png}
    \caption{Three variants of axial members which cannot maintain equilibrium}
    \label{fig:enter-label}
\end{figure}

\begin{definition}[Shear force and bending moment]
    It is then clear that equilibrium can only be maintained if at the point where the member was cut by the method of sections, a force perpendicular to the axial force and a moment are applied. These, respectively, are known as the shear force and the bending moment.
\end{definition}

\subsubsection{Arches}

Arches occur widely in structural engineering - the structural benefit of arches is that their shape can be chosen so that the internal forces in the arch are mainly in compression. This is important for stone masonry - the joints between stone blocks are typically strong in compression, but weak in tension. It can also be useful in metal structures, where reduced bending moments allow the use of lighter members.

\begin{example}
    Consider the following triangular arch, that supports a weight $W$ and that is comprised of two rigid bars, each of length $2L$, inclined at $30^\circ$ to the horizontal, pinned together at $B$ and to vertical walls at $A$ and $C$ as shown below.

    \begin{figure}[h]
        \centering
        \includegraphics{images/arch.png}
        \caption{Triangular arch example}
        \label{fig:enter-label}
    \end{figure}

    Taking moments about $A$ and by then applying vertical equilibrium, we obtain that $V_A = V_C = \frac{W}{2}$. By taking a cut about the segment $AB$ at $B$ and taking moments about $B$, we can obtain that:

    \[ 2LV_A\frac{\sqrt{3}}{2} = H_AL \iff H_A = V_A\sqrt{3} = \frac{W\sqrt{3}}{2} \]
\end{example}

Note that arches rely upon their supports, called abutments, to resist lateral movement. If abutment movement is significant, the arch may deform or even collapse.

\subsubsection{Segmented arches}

Segmented arches are those made of blocks, such as stone arches in bridges and churches. Generally, they cannot carry tensile loads at joints, and three assumptions are typically made in order to analyze them:

\begin{enumerate}
    \item The blocks are infinitely strong (no compression failure)
    \item The block interfaces are unable to carry bending moments (no tensile capacity)
    \item Block interfaces have infinite friction (there is no lateral sliding along the interfaces)
\end{enumerate}

In the simplest terms, we can deduce if a segmented arch is stable if we can draw reaction lines from the abutments to the point load, such that the lines of action of the two reactions remain within the structure.

\begin{figure}[h]
    \centering
    \includegraphics[width = 0.4\textwidth]{images/segarch1.png}
    \caption{Segmented arch under two different loads (1)}
    \label{fig:enter-label}
\end{figure}

\begin{figure}[h]
    \centering
    \includegraphics[width = 0.4\textwidth]{images/segarch2.png}
    \caption{Segmented arch under two different loads (2)}
    \label{fig:enter-label}
\end{figure}

In the above two figures, we have the same arch subjected to two different loads. In the first case, the arch is stable - we can easily draw abutment reactions such that they intersect at $W_1$. However, for the load $W_2$, the structure would collapse, because there is no way for us to draw a reaction from the right side of the arch that intersects with $W_2$, unless it goes outside the structure.

\subsubsection{Continuous arches}

Continuous (thin) arches use bending moment capacity to resist alternate loading conditions, instead of geometric thickness. These arches are always described by an equation of the type $y = y(x)$ that gives us the height of the arch at a horizontal distance $x$ away from the centre $O$.

\begin{example}
    The arch bridge illustrated below carries a uniform load $\omega$ per unit horizontal length and has parabolic shape $y = d\frac{x^2}{L^2}$. Find the abutment reactions, and the bending moment at a distance $x$ from the origin.

    \begin{figure}[h]
    \centering
    \includegraphics{images/contarch1.png}
    \caption{Pin-jointed parabolic continuous arch with uniformly distributed loading}
    \label{fig:enter-label}
    \end{figure}

    First, we need to find the abutment reactions. Let use replace the distributed load by a point load $W = 2\omega L$ at the origin. Taking moments about $B$:

    \[ 2\omega L^2 = 2V_CL \iff V_C = \omega L \]

    By vertical equilibrium, the vertical reaction at abutment $B$ is $V_B = \omega L$. Now, consider a cut at $O$ and take moments about $O$. Keep in mind that we now have a distributed load $\omega L$ acting at $\frac{L}{2}$ from the origin.

    \[ \omega \frac{L^2}{2} + Hd - \omega L^2 = 0 \iff Hd = \frac{\omega L^2}{2} \]

    Therefore, the horizontal reactions are given by:

    \[ H = \frac{\omega L^2}{2d} \]

    Now, we can consider a cut at a distance $x$ away from the origin. We can now calculate the bending moment:

    \[ M(x) + Hd\frac{x^2}{L^2} - \frac{\omega x^2}{2} = 0 \iff M(x) = \frac{\omega x^2}{2} - \frac{\omega x^2}{2} \iff M(x) = 0 \]
\end{example}

This is an interesting result - for a continuous arch with a uniformly distributed weight per horizontal length, there is no bending moment anywhere - the arch effectively behaves as if any point on it is a pin-joint. 

Furthermore, consider a continuous arch where on one side we have a force $W$, and the other side is unloaded, as in the figure below.

\begin{figure}[h]
    \centering
    \includegraphics{images/contarch2.png}
    \caption{General three-pin parabolic arch with a single point load}
    \label{fig:enter-label}
\end{figure}

On the right side $OC$, we only have the reactions at $O$ and $C$ acting - so they must be colinear. Because of this, we know that the reaction at $O$ acts parallel to $OC$. Therefore, the reaction at $B$ acts through the point of intersection between the line of action of $W$ and the line $OC$. Therefore:

\begin{enumerate}
    \item The abutment reaction at $C$ is in the direction $OC$
    \item The maximum bending moment in $OC$ is where the tangent to the arch is parallel to $OC$
    \item The maximum bending moment in $OB$ is at point $D$
\end{enumerate}

\subsubsection{Stress}

When a component made of metal or other stiff material is loaded by external forces, the component acts as a spring, changing shape under the loads. For relatively small changes from their equilibrium separation, the force between two atoms changes nearly linearly, and this is the basis of Hooke's law.

In two-dimensional problems, the state of stress has the three components illustrated in the figure below. The two direct stresses $\sigma_x$, $\sigma_y$ tend to stretch or compress the material along an axis. The shear stress $\tau_{xy}$ tends to shear the material from a small square to a rhomboid shape.

\begin{figure}[h]
    \centering
    \includegraphics{images/stress1.png}
    \caption{The three components of stress in two-dimensions}
    \label{fig:enter-label}
\end{figure}

The stresses have units of force per area, so the internal forces (stress) can be related to external loads by integrating them with respect to area, over the boundary influenced by the load. This is illustrated below for a small rectangular lamina in a state of uniform stress.

\begin{figure}[h]
    \centering
    \includegraphics{images/stress2.png}
    \caption{The relationship between external forces and the internal state of stress}
    \label{fig:enter-label}
\end{figure}

The units of stress are the same as those of pressure. However, pressure has no direction: at any point in a fluid, it acts identically in all directions. In contrast, the components of the state of stress have specific directions (as indicated by the subscripts). The state of stress must therefore always be stated with respect to a set of axes. For a given components under constant external loads, the state of stress varies as the axes rotate. 

General stress analysis is largely left to Part II courses, but the concept is central to all of structural analysis, because the state of stress is the most general description of the internal loads within a body. In this section, we will consider only conditions which lead to a uniform state of stress, as well as conditions that allows us to describe the internal state of a beam bent under the action of external loads.

\subsubsection{Thin walled shells with uniform stress}

Pressure vessels, such as balloons, pipelines, etc. are designed to carry internal or external pressure. Furthermore, we will only consider thin-walled pressure vessels, where it can be assumed that the stress distribution is uniform throughout the thickness. Also, we will consider vessels with axi-symmetric shape, such as spheres or cylinders, for which the stress distribution is uniform along any circumferential or longitudinal section.

\begin{figure}[h]
    \centering
    \includegraphics{images/stress3.png}
    \caption{Components of the state of stress for thin-walled vessels}
    \label{fig:enter-label}
\end{figure}

In the above figure, along the length (height) of the cylinder, $\sigma_z$ is the longitudinal stress, and $\sigma_\theta$ is the circumferential stress. Furthermore, $\sigma_\theta$ is the hoop stress for a spherical object.

The stress in such a pressure vessel is found using the method of sections. As with the analysis of trusses and beams, a cut is made around one section of the loaded pressure vessel, typically a plane-cut through the vessel in some direction. Asserting equilibrium across this plane, the external force must equal the internal force found as the stress in the wall of the pressure vessel acting normal to the plane multiplied by the area of the cut surface of the vessel, i.e.:

\[ pS_{\text{projected}} = \sigma_{\text{normal}}S_{\text{cut}} \]

Using the convention that tensile stress is positive.

\begin{example}[Hoop stress in a sphere]
    Consider a sphere with radius $R$ and thickness $t$.

    \begin{figure}[h]
        \centering
        \includegraphics[width = 0.15\textwidth]{images/stress4.png}
        \caption{A free-body diagram for half a pressured spherical balloon}
        \label{fig:enter-label}
    \end{figure}

    Therefore:

    \[ \pi pR^2 = \sigma_\theta 2\pi Rt \iff \sigma_\theta = \frac{pR}{2t} \]

    This is the hoop stress in a sphere pressure vessel.
\end{example}

\begin{example}[Circumferential stress in a cylinder vessel]
    Consider a cylinder of radius $R$ and thickness $t$, as shown below.

    \begin{figure}[h]
        \centering
        \includegraphics[width = 0.45\textwidth]{images/stress5.png}
        \caption{A long cylinder subject to internal pressure}
        \label{fig:enter-label}
    \end{figure}

    We can apply the identity above again:

    \[ 2pRL = 2\sigma_\theta Lt \iff \sigma_\theta = \frac{pR}{t}  \]
\end{example}

\begin{proposition}[Two-dimensional stress equations]
    For a two-dimensional object, we can deduce that the state of stress obeys the following relationships:

    \[ \frac{\partial \sigma_x}{\partial x} + \frac{\partial \tau_{xy}}{\partial y} + b_x = 0 \]
    \[ \frac{\partial \sigma_y}{\partial y} + \frac{\partial \tau_{xy}}{\partial x} + b_y = 0 \]
\end{proposition}

\newpage

\subsection{Deflection}

Having now explored the equilibrium of external forces and their translation into internal forces within a structure, we can now begin to consider how the structure deforms. Although structures may fail due to limits to their strength, many designs are limited more by deflection. 

\subsubsection{Cables and bar extensions}

Cables, strings, ropes and chains are able to carry uniaxial tension, but they are unable to carry compression or bending, as they have negligible bending stiffness. As a result, it is a good assumption that the tension at any point acts in the direction of the tangent to the cable at that point.

\begin{example}
    Consider the following structure below. Our goal is to find a relationship between $W_1$, $W_2$ and $\alpha$.

    \begin{figure}[h]
        \centering
        \includegraphics{images/cable1.png}
        \caption{Rope bridge supported by pulley and weight}
        \label{fig:enter-label}
    \end{figure}

    For the right side of the structure, $T_{\text{BC}} = W_2$. Now, by means of a force polygon, since $T_{\text{AB}} = T_{\text{BC}}$, we deduce that:

    \[ \sin{\alpha} = \frac{W_1}{2W_2} \]
\end{example}

\subsubsection{Shape of a cable subject to distributed loads}

Consider a shallow cable with mid-point displacement $d << L$. We will now derive the shape of the cable by making use of two methods. Let us suppose that the load has a weight distribution of $\omega$ N/m.

\begin{figure}[h]
    \centering
    \includegraphics[width = 0.5\textwidth]{images/cable2.png}
    \caption{Cable subject to uniformly distributed load}
    \label{fig:enter-label}
\end{figure}

By taking moments about $A$ and then by means of a vertical equilibrium arguments, we deduce that $V_A = V_C = \omega L$.

Now, consider a cut through $B$ and take moments about point $B$:

\[ V_AL = Hd + \frac{\omega L^2}{2} \]

This is equivalent to:

\[ \omega L^2 = Hd + \frac{\omega L^2}{2} \iff Hd = \frac{\omega L^2}{2} \iff H = \frac{\omega L^2}{2d} \]

Now, consider a cut at a random distance $x$ from point $B$. By taking moments about $X$:

\[ \omega x \frac{x}{2}  = Hy \iff \frac{\omega x^2}{2} = \frac{\omega L^2}{2d}y \]

Therefore, we obtain that:

\[ y(x) = d\left(\frac{x}{L}\right)^2 \]

Note that we measure $y$ with respect to $B$. In general, we consider the origin as the point where the cable sags the most. We always have to consider the cut with respect to the origin.

Now, we can also consider the forces on a small elements of cable at a distance $x$ to the right and $y$ above the centre $B$. By horizontal equilibrium:

\[ H = H + \delta H \iff \delta H = 0 \]

By vertical equilibrium:

\[ V + \delta V = V + \omega \delta x \iff \delta V = \omega\delta x \iff \frac{\delta V}{\delta x} = \omega \]

Furthermore, $\frac{\delta y}{\delta x} = \frac{V}{H}$. By taking both of these into the limit, we deduce that $\frac{dV}{dx} = \omega$ and $\frac{dy}{dx} = \frac{V}{H} \iff V = H\frac{dy}{dx}$. By differentiating this:

\[ \frac{dV}{dx} = H\frac{d^2y}{dx^2} \iff \omega = H\frac{d^2y}{dx^2} \]

Therefore:

\[ \frac{d^2y}{dx^2} = \frac{\omega}{H} \]

By solving this differential equation, and setting the boundary conditions as $y(0) = 0$ and $\frac{dy}{dx} = 0$ at $x = 0$, we can deduce the same equation as before. 

\begin{proposition}[Length of a cable]
    Suppose we have a cable of shape $y = y(x)$ that spans over a horizontal length $L$. Then, the total length of the cable is given by:

    \[ l = \int_0^L \sqrt{1 + \left(\frac{dy}{dx}\right)^2}dx \]
\end{proposition}

Also note that the tension in the string at any given point is $T = \sqrt{T_x^2 + T_y^2}$.

\subsubsection{Strains, Hooke's law and bar extensions}

The Young's Modulus $(E)$ is a material property, and the relationship that governs the force and the extension is given by:

\[ F = \frac{ES_0}{L_0}\Delta L \]

\begin{definition}[Stress]
    We define the stress as $\sigma = \frac{F}{S}$. This is the same as previously defined in the section where we analyzed pressure vessels.
\end{definition}

\begin{definition}[Strain]
    The strain (in one dimension) is the relative elongation of the member, i.e.:

    \[ \epsilon = \frac{\Delta L}{L_0} \]
\end{definition}

Furthermore, Hooke's law  can be written as:

\[ \sigma = E\epsilon \]

Also, the extension of any member can be calculated as:

\[ e = \frac{FL_0}{ES_0} \]

Moreover, members can extend (and contract) due to temperature. Experiments have shown that the one-dimensional thermal strain $\epsilon_T$ in a long, straight, slender member due to a uniform temperature change $\Delta\theta$ can be predicted using the formula:

\[ \epsilon_T = \alpha\Delta\theta \]

Therefore, the total strain in a member is:

\[ \epsilon = \frac{F}{ES_0} + \alpha\Delta\theta \]

And lastly, we can deduce that the total extension experienced by the member is:

\[ e = \frac{FL_0}{ES_0} + \alpha\Delta\theta L_0 \]

\subsubsection{Displacement diagrams}

We will now analyze what happens when all members of a structure change length. However, we make two assumptions: the extensions are very small in comparison to the length of each member. Therefore, we assume that:

\begin{enumerate}
    \item All extensions are parallel to the original member
    \item All rotations are perpendicular to the original member
\end{enumerate}

Consider a general truss structure. The general procedure to draw a displacement diagram and deduce the extensions of each member is:

\begin{enumerate}
    \item Select the origin pole of the displacement diagram so that the displacement of all other points can be measured relative to $O$; note that the origin must be a point (or multiple points) that do not change position
    \item Draw extensions for members $P_iP_k$ and $P_jP_k$
    \item Draw the rotations for the above members
    \item Intersect them in order to deduce the position of point $P_k$
\end{enumerate}

In general, when given a truss structure, we must first calculate all the extensions my first computing the bar forces and then using Hooke's law. Afterwards, we can draw the displacement diagram. Consider the below truss structure, with the shown extensions.

\begin{figure}[h]
    \centering
    \includegraphics[width = 0.5\textwidth]{images/displacement1.png}
    \caption{Truss structure with displacements}
    \label{fig:enter-label}
\end{figure}

It is obvious that in this case points $D$ and $C$ do not move. Therefore, we choose them as the origin. Now, we can draw the following displacement diagram:

\begin{figure}[h]
    \centering
    \includegraphics[width = 0.25\textwidth]{images/displacement2.png}
    \caption{Displacement diagram}
    \label{fig:enter-label}
\end{figure}

\subsubsection{Displacement for symmetric structures}

Consider the following structure:

\begin{figure}[h]
    \centering
    \includegraphics{images/displacement3.png}
    \caption{Symmetric structure}
    \label{fig:enter-label}
\end{figure}

As we can observe, the above structure is symmetric about the $DB$ vertical member. For this reason, $DB$ needs to remain vertical - it cannot rotate. For this reason, we can choose $D$ as the origin of our displacement diagram, and immediately draw the extension of member $DB$ to find $B$ vertically above $D$.

\subsubsection{Real work}

The principle of conservation of energy states that in a closed system, the total energy is conserved over time. The experiment of a mass hung on a wire is now repeated. Considering this as a closed system, when the weight is applied to the wire, the potential energy lost by the weight is taken up as energy stored in the wire due to its stretching, and any kinetic energy due to the weight moving.

\begin{proposition}[Real work for plane trusses]
    Consider a truss structure with forces $\mathbf{F}_i$, $1 \leq i \leq n$ applied to it, leading to displacements $\mathbf{d}_i$, tensions $\mathbf{T}_j$ and extensions $\mathbf{e}_j$. Hence:

    \[ \sum_i \mathbf{F}_i \cdot \mathbf{d}_i = \sum_j \mathbf{T}_j \cdot \mathbf{e}_j \]

    And because the extensions in the bars are parallel to each member (and thus parallel to each tension):

    \[ \sum_i \mathbf{F}_i \cdot \mathbf{d}_i = \sum_j T_je_j \]
\end{proposition}

For instance, if we would apply a unit load of size $1$ (this is just to simplify our calculations, as a load of size $1$ actually means $W$) at point $D$ in the truss above, we can deduce that:

\[ \delta_{D}^V = 2.9 \frac{WL}{AE} \]

Note that real work is applied when we want to calculate the displacement of a point in the direction of the force applied to it. If there is no force applied to it, however, we must use virtual work.

\subsubsection{Virtual work}

\begin{proposition}[Virtual work]
    Suppose we wish to find the displacement of a joint $P_i$ due to a load applied at joint $P_j$. To do this, we will first compute all the real extensions imposed by the load at joints $P_j$. Afterwards, we impose a unit load in the direction (either horizontal or vertical) of the displacement, in order to find the virtual set of tensions $T^*_j$. Therefore:

    \[ \sum_i \mathbf{F}_i^* \cdot \mathbf{d}_i = \sum_j T_j^* e_j \]

    The stars indicate that we have a set of virtual forces and tensions, and a set of real extensions and displacements. We may also utilize a real set of forces and impose a solid body rotation of an angle $\psi$ in order to deduce the force in certain members. However, this is not the most effective way to calculate forces, as all it does is that it just mimics moments with an extra headache added. We therefore use virtual work to determine the extensions of certain joints, and the method of sections/method of joints if we wish to determine the forces in certain members.
\end{proposition}

\subsubsection{Structural optimisation}

Suppose we have a plane structure and a pin-joint $P_k$ - we want to reduce the deflection (horizontal or vertical) under this loading by adding some mass (i.e., volume) to one of the bars - which one do we choose? By means of virtual work, we know that:

\[ \delta_{P_k} = \sum_j T_j^* \frac{T_jL_j}{A_jE} \]

However, we note that $V_j = A_jL_j \iff \delta_{P_k} = \sum_j T_j^* \frac{T_jL_j^2}{V_jE}$ is the displacement of point $P_k$ under this loading.

Let us fix a random member $j$ and take the partial derivative of the above expression with respect to the volume of the bar:

\[ \frac{\partial P_k}{\partial V_j} = -T_j^* \frac{T_jL_j^2}{V_j^2E} = -T_j^* \frac{T_j}{A_j^2E}  \]

Therefore, the bar for which the deflection would be minimized if we added mass to it is the bar for $j, 1 \leq j \leq n$ for which $\frac{\partial P_k}{\partial V_j}$ is minimized (or the most negative). Conversely, the bar for which the same partial derivative is maximized, the deflection is the biggest.

\newpage

\section{Materials}

Though they lacked the means to prove it, the ancient Greeks suspected that solids were made of discrete
atoms that packed in a regular, orderly way to give crystals. With modern techniques of X-ray and
electron diffraction and high-resolution microscopy, we know that all solids are indeed made up of
atoms or molecules, and that most (but not all) are crystalline. Most engineering metals and
ceramics are made up of many small crystals, or grains, stuck together at grain boundaries to make
polycrystalline microstructures.

\subsection{Atoms, solutions and compounds}

Atoms consist of a nucleus of protons (positive charge) and neutrons, with different elements defined by the number of protons in the nucleus. Electrons (negative charge) orbit the nucleus to balance the proton charge in discrete shells of fixed energy levels. 

\begin{definition}[The atomic number]
    For an element $X$, we define the atomic number $Z$ as the number of electrons (or protons) in the atom.
\end{definition}

\begin{definition}[The atomic weight]
    For an element $X$, we define the atomic weight $A$ to be the mass of the nucleus, which is also equal to the number of protons and neutrons.
\end{definition}

In general, we denote an element $X$ as ${}_Z X^A$. For example, for lead, we write this as ${}_{82} Pb^{207}$.

\subsubsection{Atomic size}

A surprising and important feature is that all atoms are a similar size, i.e. atomic radii are all of order
$0.1–0.2$ nm, while the atomic weight spans a factor of over $200$. In simple terms, this is because as the
number of protons and electrons increase, the electron shells are drawn in to smaller radii. Conversely, the are consequences to this:

\begin{enumerate}
    \item Mixtures of atoms of different elements (alloys) can pack efficiently into crystal lattice structures, forming solid solutions or compounds
    \item Most solid solutions will be substitutional - atoms of similar size replace one another in the lattice, e.g. $\ch{CuZn}$ in brass
    \item Only small atoms (e.g. $\ch{H}$, $\ch{C}$) form interstitial solid solutions - these atoms can fit into the gaps between metal atoms
    \item Compounds can form readily, with lattices that satisfy the required atomic fractions of the elements, or stoichiometry (e.g., $\ch{Fe3C}$, $\ch{Al2O3}$)
\end{enumerate}

\newpage

\subsection{Atomic and molecular bonding}

Atomic bonding is determined by the interaction between the outermost electrons in atoms. There are two types of bonds:

\begin{enumerate}
    \item Primary bonds: metals, ceramics, and along long-chain polymer molecules
    \item Secondary bonds: between polymer chains, and in materials such as ice
\end{enumerate}

Note that primary bonds are 100 times stronger than secondary bonds, and hence more difficult to stretch and break.

\subsubsection{Primary bonding}

\begin{definition}[Metallic bonding]
    In metallic bonding, the atoms form positively charged ions by releasing a few electrons, which then form a sea of free electrons. Bonding is done by electrostatic interaction between the ions an free electrons.

    \begin{enumerate}
        \item The bonds are equally strong in all directions - the metallic bond is non-directional
        \item Metallically bonded compounds form regular crystal lattices
        \item The free electrons are not bound to specific atoms, hence metalically bonded materials are electrical conductors
    \end{enumerate}
\end{definition}

\begin{definition}[Ionic bonding]
    In ionic bonding, electrons are transferred permanently between atoms to produce stable, oppositely charged ions, which then attract electrostatically.

    \begin{enumerate}
        \item The electrostatic forces are equal in all directions - the ionic bond is non-directional
        \item Positive and negative ions pack into regular crystal lattice structures
        \item Electrons are bound to specific ions, so ionically bounded materials are electrical insulators
    \end{enumerate}
\end{definition}

\begin{definition}[Covalent bonding]
    In covalent bonding, the electrons are shared between atoms to achieve an energetically stable number.

    \begin{enumerate}
        \item The shared electrons are associated with particular electron shells of the bonding atoms, and hence the covalent bond is directional
        \item Covalently bonded materials form regular crystal lattices (e.g. diamond), networks (e.g. glasses) or long-chain molecules (e.g. polymers)
        \item The electrons are bound to specific atoms, so covalently bounded materials are also electrical insulators
    \end{enumerate}
\end{definition}

\subsubsection{Modelling primary bonds}

Primary bonds may be modelled as stiff springs between the atoms (or ions) with a non-linear force-separation characteristic. The atoms (or ions) have an equilibrium separation $r_0$, governed by the balance between attractive and repulsive forces. At the dissociation separation, the atoms (or ions) can be separated completely. 

Atoms vibrate about the equilibrium separation with kinetic energy approximately equal to $k_BT$, where $k_B \approx 1.38 \times 10^{23}$ J/K is the Boltzmann constant, and $T$ is the absolute temperature. 

All bonds effectively break down when $k_BT$ exceeds the bond energy - at this point the material melts. Due to the strength of their primary bonds, metals and ceramics have a characteristically high melting temperature.

\begin{figure}[ht!]
    \centering
    \includegraphics[width = 0.5\textwidth]{images/bond1.png}
    \caption{Force-separation characteristic}
    \label{fig:enter-label}
\end{figure}

\subsubsection{Secondary bonding}

Secondary, or Van der Waals bonds, operate at much larger atomic separation than primary bonds, and hence are much weaker. They are associated with dipoles - molecules in which the centres of positive and negative charge do not coincide.

Note that Van der Waals bonds between hydrogen atoms form the strongest dipoles and are the commonest secondary bond between polymer chains. Moreover, in secondary bonded materials (such as polymers), the bonds become ineffective at much lower thermal energies than primary bonds, giving low melting points.

\newpage

\subsection{Metal crystal structures}

Primary bonding gives a well-characterised equilibrium spacing, with stiff restoring forces. For the
purposes of packing, the atoms may be treated as hard spheres, forming a solid crystal lattice.

The great majority of the 92 stable elements are metallic, and of these, the majority (68 in all) have one of
just three simple structures: 

\begin{enumerate}
    \item Face-centred cubic (FCC)
    \item Close-packed hexagonal (CPH)
    \item Body-centred cubic (BCC)
\end{enumerate}

In total, there are 14 distinguishable three-dimensional crystal lattices, but these three structures are all that is needed for most engineering purposes.

\subsubsection{Close-packed crystal structures}

The basic building block for the first two of these structures is the
close-packed plane (i.e. the highest density of atoms arranged in a
plane is a hexagonal packing).
The close-packed directions are the straight lines through the centres
of touching atoms (there are 3 in a close-packed plane).

\begin{figure}[h]
    \centering
    \includegraphics{images/mat1.png}
    \caption{Close-packed plane}
    \label{fig:enter-label}
\end{figure}

A 3D lattice can be built by stacking these close-packed planes – but there are 2 ways of doing it.
Around each atom there are 6 locations in which atoms can sit.
But only 3 of these can be occupied at once (“odd” or “even”).
Imagine placing a first layer above the reference layer (in either odd
or even locations). 

\begin{figure}[h]
    \centering
    \includegraphics{images/mat2.png}
    \caption{The possible positions of atoms}
    \label{fig:enter-label}
\end{figure}

There are then two options for placing a layer
below: one using the same locations with respect to the initial layer
(giving ABA stacking), the other using the alternative locations
(giving ABC stacking), i.e. in side view:

\begin{figure}[h]
    \centering
    \includegraphics{images/mat3.png}
    \caption{Possible stacking for crystal lattices}
    \label{fig:enter-label}
\end{figure}

In fact, $ABC$ stacking is simply the face-centred cubic, and the $ABA$ stacking is the close-packed hexagonal. The difference is seen more clearly from their unit cells, which we will discuss just below. Both of these structures are close-packed, i.e. the spheres occupy as large a fraction of the volume as possible.

The apparently minor packing difference is of little consequence for elastic properties, but has a big influence on plastic deformation.

\subsubsection{Unit cells}

\begin{definition}[Unit cell]
    A unit cell is the smallest unit which can be replicated by translation in all directions to build up the three-dimensional crystal structure. The unit cells dimensions are called the lattice constants.
\end{definition}

Note that the unit cells are drawn with the atoms reduced in size, for clarity - remember that they touch in the close-packed directions.

\begin{definition}[The face-centred cubic (FCC) structure]
    We define the FCC structure as a cubic unit cell with one atom at each corner and one at the centre of each face. Any diagonal of any face is a close-packed direction, and the lattice constant is the length of any side of the cube.
\end{definition}

\begin{figure}[h]
    \centering
    \includegraphics[width = 0.45\textwidth]{images/mat4.png}
    \caption{Face-centred cubic structure of packed spheres showing the $ABC$ stacking}
    \label{fig:enter-label}
\end{figure}

\begin{proposition}
    For the FCC structure, the ratio of the lattice constant to the atomic radius is $2\sqrt{2}$.
\end{proposition}

\begin{proof}
    On the diagonal of any face of the FCC structure, there is one full atom and two halves. Therefore, since this is a closed-packed direction, this means that the length of the diagonal of any face of the cube is $4R$, where $R$ is the atomic radius. By Pythagoras' theorem:

    \[ a^2 + a^2 = (4R)^2 \iff 2a^2 = 16R^2 \iff \frac{a}{R} = 2\sqrt{2} \]
\end{proof}

FCC metals have the following characteristics:

\begin{enumerate}
    \item They are very ductile when pure, work hardening rapidly but softening again when annealed, allowing them to be rolled, forged, drawn or otherwise shaped by deformation processing
    \item They are generally though, i.e. resistant to crack propagation
    \item They retain their ductility and toughness t o absolute zero, something very few other materials allow for
\end{enumerate}

\begin{definition}[Close-packed hexagonal (CPH) structure]
    We define the CPH structure as a prismatic hexagonal unit cell with an atom at each corner, one at the centre of the hexagonal faces, and three in the middle. Note that this is exactly an $ABA$ type of stacking, and that the close-packed directions are the sides of the hexagonal faces.
\end{definition}

Moreover, there are two separate lattice constants in CPH - the side-length of the hexagonal base, $a$, and the height of the prism, $c$. 

\begin{figure}[h]
    \centering
    \includegraphics{images/mat6.png}
    \caption{CPH structure of packed spheres showing the $ABA$ stacking}
    \label{fig:enter-label}
\end{figure}

\begin{proposition}
    For the CPH structure, the ratio of the lattice constants $\frac{c}{a}$ is equal to $1.633$.
\end{proposition}

\begin{proof}
    Because the hexagonal sides are the close-packed directions, and because we have two halves of atoms per side, this means that $a = 2R$. Now, since two consecutive side atoms, with the middle atom and an atom at the half height form a tetrahedron, the height of this tetrahedron is just $\frac{c}{2}$. The side of the tetrahedron is then $a = 2R$, and the height of any of its faces is $h = \frac{a\sqrt{3}}{2}$. Because the height of the tetrahedron passes through the centre of mass of the opposite equilateral triangle, we can apply pythagoras do determine the height of the tetrahedron:

    \[ H^2 = h^2 - \left(\frac{a\sqrt{3}}{6}\right)^2 \iff H^2 = \frac{3a^2}{4} - \frac{3a^2}{36} = \frac{24a^2}{36} = \frac{2a^2}{3}\]

    Hence, the height of the tetrahedron is:

    \[ H = \frac{a\sqrt{2}}{\sqrt{3}} \iff c = \frac{2a\sqrt{2}}{\sqrt{3}} \]

    And then, $\frac{c}{a} \approx 1.633$, thus concluding our proof.
\end{proof}

CPH metals have the following characteristics:

\begin{enumerate}
    \item They are reasonably ductile (at least when hot), allowing them to be forged, rolled, and drawn, but in a more limited way than FCC metals
    \item Their structure makes them more anisotropic than FCC metals (i.e. crystal properties vary with direction)
\end{enumerate}

\begin{definition}[Body-centred cubic (BCC) structure]
    We define the BCC structure as a cubic unit cell with one atom at each corner and one in the middle of the cube. Note that this structure is not close-packed - it is made by stacking planes of atoms in a square array (not hexagonal). Also, the close-packed directions are the cube diagonals, and the lattice constant is the cube's side.
\end{definition}
    
\begin{proposition}
    For the BCC structure, the ratio of the lattice constant to the atomic radius is $\frac{4R}{\sqrt{3}}$. 
\end{proposition}

\begin{proof}
    Since the cube diagonal is the close-packed direction and because its length is $4R$, we can apply Pythagoras' theorem:

    \[ (4R)^2 = a^2 + (a\sqrt{2})^2 = 3a^2 \iff 16R^2 = 3a^2 \]

    Therefore:

    \[ \frac{a}{R} = \frac{4R}{\sqrt{3}} \]
\end{proof}

BCC metals have the following characteristics:

\begin{enumerate}
    \item They are ductile, particularly when hot, allowing them to be rolled, forged, drawn or otherwise shaped by deformation processing
    \item They are generally tough, and resistant to crack propagation
    \item They become brittle at low temperatures. The change happens at the “ductile-brittle transition temperature”, limiting their use below this
\end{enumerate}

\subsubsection{Grain structure}

Metal components are commonly manufactured by casting – solidification of a liquid poured into a shaped mould. The solidification mechanism involves the formation of many solid crystalline nuclei, which grow by attachment of atoms to the crystal at the interface between liquid and solid. This is explored further in the IB Materials course. For now, we note that solidification is completed when adjacent crystals impinge on one another. But because the orientation of the packing in each crystal is random, there is a misfit in the atomic packing at the interface between two crystals. The individual crystals are called grains, and the region of imperfect packing is called a grain boundary – see the figure below.

\begin{figure}[h]
    \centering
    \includegraphics[width = 0.35\textwidth]{images/mat7.png}
    \caption{Grain boundary illustration}
    \label{fig:enter-label}
\end{figure}

\newpage

\subsection{Theoretical density of metals}

We will now introduce a way of calculating the theoretical density of any metal.

\subsubsection{Atomic packing fraction}

\begin{definition}[Atomic packing fraction]
    We define the atomic packing fraction as the fraction of space occupied by the atoms in an unit cell.
\end{definition}

\begin{example}
    Let us determine the atomic packing fraction of the FCC structure. We know that $a = 2\sqrt{2}R$, and therefore the total volume of the unit cell is:

    \[ V = a^3 = 16\sqrt{2}R^3 \]

    In total, we have $8 \times \frac{1}{8} + 6 \times \frac{1}{2} = 4$ atoms per unit cell. Therefore, the volume occupied by the atoms is:

    \[ V_0 = 4 \times \frac{4\pi R^3}{3} \]

    Hence, the atomic packing fraction is then:

    \[ f = \frac{V_0}{V} \approx 0.74 \]
\end{example}

Note that for the CPH, the atomic packing fraction is the same as for the FCC, however for the BCC it is lower, proving that the BCC is not a close-packed structure.

\subsubsection{Evaluation of theoretical density}

The density of crystalline materials depends directly on the number of atoms per unit volume, and the atomic mass of the atoms.

\begin{proposition}[Theoretical density]
    For a unit cell with $i$ types of different atoms with $n_i$ atoms of each type and with atomic mass $A_i$, the theoretical density is:

    \[ \rho = \frac{\sum_i n_iA_i}{V_cN_A} \]

    Where $V_c$ is the volume of the cell and $N_A$ is Avogadro's number, i.e. $N_A \approx 6.02 \times 10^{23}$ mol$^{-1}$.
\end{proposition}

\newpage

\subsection{Interstitial space}

\begin{definition}[Interstitial space]
    An interstitial space (or hole) is the space between the atoms or molecules. The FCC, CPH and BCC structures contain interstitial space of two sorts: tetrahedral and octahedral. These are defined by the arrangement of the surrounding atoms, as shown in the figures below.
\end{definition}

\begin{figure}[h]
    \centering
    \includegraphics[width = \textwidth]{images/mat8.png}
    \caption{Interstitial holes in various unit cell structures}
    \label{fig:enter-label}
\end{figure}

Interstitial holes are important because small foreign atoms can fit into them. For FCC and CPH structures, the tetrahedral hole can accommodate, without strain, a sphere with a radius of 0.22 of that of the host. The octahedral holes are almost twice as large. Atoms are in reality somewhat elastic, so that foreign atoms that are larger than the holes can be squeezed into the interstitial space. Note that to find the diameter of the sphere, we simply calculate the distance between two base atoms and subtract the length they occupy (i.e. $2R$).

Interstitial solute atoms are particularly important for carbon steel, which is iron with carbon in some of the interstitial holes. Iron is BCC (at room temperature), and only contains tetrahedral holes (shown in Figure (c) above). These can hold a sphere with a radius 0.29 times that of the host, without strain. Carbon will go into these holes, but because it is a bit too big, it distorts the crystal structure. It is this distortion that gives carbon steels much of their strength. 

Another significant factor in carbon steels is the difference in maximum hole size between FCC and BCC. Iron transforms to FCC (at temperatures around $800^\circ$C, depending on the C content). This means that much more carbon will “dissolve” in FCC (at high temperature) than in BCC (at room temperature) – this is central to the heat treatment and strengthening of carbon steel (covered in Part IB). Interstitial holes appear in another context below: they give a way of understanding the structures of many ceramic compounds: oxides, carbides and nitrides.

\begin{example}
    Compute the diameter of the largest sphere that will fit in the octahedral hole in the FCC structure.

    To do so, we first recall that the close-packed direction is the cube face diagonal. Therefore, the diagonal is $4R$, and hence the side of the FCC is just $a = 2\sqrt{2}R$. This is also the base length in the octahedral hole. Therefore, the diameter is just:

    \[ d = 2\sqrt{2}R - 2R = 2(\sqrt{2} - 1) \]
\end{example}

\newpage

\subsection{Ceramic crystals}

Technical ceramics are the hardest, most refractory structural materials. The ceramic family also includes many functional materials (semiconductors, piezoelectrics, ferromagnetic etc.). Their structures often look complicated, but can mostly be interpreted as atoms of one type, arranged on a simple FCC, CPH or BCC lattice, with the atoms of the second type (and sometimes a third) inserted into the interstitial holes of the first lattice. There are four main crystal structures for engineering ceramics: diamond cubic, halite, corundum and fluorite. Two of these are illustrated here, to demonstrate the underlying principles.

\subsubsection{The diamond-cubic (DC) structure}

The hardest ceramic of all is diamond, of major importance for cutting tools, polishes and scratch-resistant coatings. Silicon and germanium, the foundation of semiconductor technology, have the same structure. Carbon, silicon and germanium atoms have a 4-valent nature – each atom prefers to have 4 nearest neighbours, symmetrically placed around them. The DC structure achieves this. 

The figure below shows the DC unit cell. If you first ignore the numbered atoms, the remainder form an FCC lattice; the atoms numbered 1-4 are then additional atoms located in half of the tetrahedral interstitial spaces. As the tetrahedral hole is far too small to accommodate a full-sized atom, the others are pushed further apart, lowering the density.

\begin{figure}[h]
    \centering
    \includegraphics[width = 0.25\textwidth]{images/mat9.png}
    \caption{The diamond-cubic structure}
    \label{fig:enter-label}
\end{figure}

Silicon carbide (like diamond) is very hard, and its structure is closely related. Carbon lies directly above silicon in the periodic table, it has the same crystal structure and is chemically similar. So it is no surprise that silicon carbide, with the formula \ch{SiC}, has the diamond structure with half the carbon atoms replaced by silicon. 

\begin{figure}[h]
    \centering
    \includegraphics[width = 0.25\textwidth]{images/mat10.png}
    \caption{The structure of silicon carbide}
    \label{fig:enter-label}
\end{figure}

\subsubsection{Oxides with the Corundum structure}

A number of oxides have the chemical formula $\ch{M2O3}$, among them alumina, $\ch{Al2O3}$. The oxygen, the larger of the two ions, is close-packed in a CPH stacking, and the M atoms occupy two thirds of the octahedral holes in the lattice.

\begin{figure}[h]
    \centering
    \includegraphics{images/mat11.png}
    \caption{The M atoms of the corundum structure}
    \label{fig:enter-label}
\end{figure}

\subsubsection{Glasses}

When crystalline materials melt, the atoms lose their regular packing but are still loosely held together; on solidification, crystals usually form readily. Glasses are all based on silica, SiO2, for which crystallisation is difficult. In the solid state silica usually has an amorphous (or glassy) structure, and only crystallises if cooled very slowly. The difference is shown schematically in two dimensions below.

\begin{figure}[h]
    \centering
    \includegraphics{images/mat12.png}
    \caption{Glass structures}
    \label{fig:enter-label}
\end{figure}

Amorphous structures give transparency, with the colour and refractive index of the glass readily
being customised by alloying.

\newpage

\subsection{Polymer microstructure}

Polymers are long-chain molecules of carbon (typically $10^4$ – $10^6$ atoms). Along the chains are side bonds to atoms of H, Cl, F, or groups of atoms such as a methyl group, $\ch{CH3}$. The simplest polymer (polyethylene) is formed by polymerisation of a basic $\ch{CH2}$ monomer into a chain molecule.

Primary bonding between the C atoms is by strong covalent bonds – both along the chains, and at cross-links (where two chains are bonded together). Secondary bonding acts between the chains (via the side-groups) by weak Van der Waals bonds.

\begin{figure}[h]
    \centering
    \includegraphics[width = 1\textwidth]{images/mat13.png}
    \caption{Polymer bonds}
    \label{fig:enter-label}
\end{figure}

Polymers are inherently low in density (similar to that of water): they are made of light elements (carbon, hydrogen), and the low packing density of the molecules leaves more “free space” in the structure

\subsubsection{Microstructure in polymers}

There are three main classes of polymer: thermoplastics, thermosets and elastomers. In all cases the long-chain molecules pack together randomly, giving an amorphous “spaghetti-like” microstructure. The classes are then distinguished by the detail in the molecular architecture, in particular, whether the extent of covalent cross-linking between chains.

\begin{definition}[Thermoplastics]
    Thermoplastics contain no cross-links (covalent bonds between the chain molecules), but are divided into two sub-groups: amorphous and semi-crystalline.
\end{definition}

\begin{figure}[h]
    \centering
    \includegraphics{images/mat14.png}
    \caption{Thermoplastics}
    \label{fig:enter-label}
\end{figure}

In amorphous thermoplastics, the long-chain molecules are arranged entirely at random, with occasional entanglement points between chains. At these points there is no additional bonding, but they do restrain the deformation and sliding of the molecules. Semi-crystalline thermoplastics are partly amorphous, and partly ordered in crystalline regions (known as “spherulites”).

\begin{definition}[Elastomers and thermosets]
    Elastomers contain a small number of cross-links, between simple chain molecules. Natural rubber is an example, in which the cross-links are provided by sulphur. Further cross-linking can be triggered in service (e.g. by UV light or ozone), leading to polymer degradation. Thermosets, in contrast, have extensive cross-links between chains.
\end{definition}

\begin{figure}[h]
    \centering
    \includegraphics{images/mat15.png}
    \caption{Elastomers and thermosets}
    \label{fig:enter-label}
\end{figure}

\end{document}
